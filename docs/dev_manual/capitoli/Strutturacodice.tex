\section{Struttura del codice sorgente}\label{sec:Struttura_Codice}
Al fine di fornire un codice ben organizzato, data la media dimensione in termini di righe di codice del software sviluppato, si è deciso dal principio di strutturarlo in \textbf{moduli} e \textbf{sottomoduli}. Il seguente schema ne mostra l'organizzazione.\\

\large{
    \dirtree{%
    .1 src/.
    .2 hardened \ldots\ldots\ldots\ldots\ldots\ldots{} \begin{minipage}[t]{10cm}
                            Definizione e implementazione del tipo \texttt{Hardened<T>}; definizione del tipo enumerativo \texttt{IncoherenceErorr}\\
                          \end{minipage} . 
    .2 fault\_list\_manager \ldots\ldots\ldots{} \begin{minipage}[t] {10 cm}
        Generazione della fault list\\
    \end{minipage}.
    .3 static\_analysis \ldots\ldots\ldots{} \begin{minipage}[t]{10cm}
          Analisi statica automatica del codice\\
    \end{minipage}.
    .2 injector \ldots\ldots\ldots\ldots\ldots\ldots{} \begin{minipage}[t]{10cm}
        Iniettore dei fault\\
    \end{minipage}.
    .3 algorithms \ldots\ldots\ldots\ldots\quad{} \begin{minipage}[t]{10cm}
        Adattamento del codice dei tre casi di studio alla fault injection software\\
    \end{minipage}. 
    .2 analizer \ldots\ldots\ldots\ldots\ldots\ldots{} \begin{minipage}[t]{10cm}
        Calcoli di aggregati e statistiche, analisi dei risultati, generazione di report in pdf
    \end{minipage}.
}
}
