\section{Introduzione}
Nella società attuale si fa, in generale, un utilizzo capillare di sistemi computerizzati, questi nello specifico sono coinvolti in settori in cui un guasto al sistema potrebbe essere critico, mettendo potenzialmente a rischio vite umane. L'implementazione di sistemi \textit{safety-critical} espone lo sviluppatore ad affrontare problemi non trascurabili che coinvolgono la valutazione della \textbf{dependability} \cite{noauthor_dependability_2024} e della \textbf{tolleranza ai guasti} (\textit{fault tolerance}). Le tecniche di test standard e l'utilizzo di benchmark non bastano in quest'ambito, in quanto per valutare certi aspetti del sistema (quali la dependability) bisognerebbe osservarne il comportamento dopo che il guasto si verifica. \\
In ambito 'fault-tolerance' vengono utilizzate delle metriche specifiche per quantificare affidabilità e robustezza del sistema in analisi. Per citarne una significativa, riportiamo l'\textit{MTBF} (Mean Time Between Failure); questa per i sistemi concepiti, per essere tolleranti ai guasti, potrebbe essere associata ad un periodo di tempo molto lungo, anche anni! Da tempo la ricerca va nella direzione di trovare un modo per 'accelerare' in simulazione l'occorrenza di questi guasti/difetti prima che accadano naturalmente, dal momento che questa lunga latenza rende difficile anche solo identificare la causa di un potenziale difetto/guasto. In molti casi l'approccio utilizzato è quello basato su esperimenti di \textbf{fault injection}, che -- come riportato in \cite{depend} -- vengono usati non solo durante l'implementazione, ma anche durante la fase prototipale e operativa, permettendo quindi di coprire un ampio spettro di casistiche.\\
La survey \cite{hsueh1997fault} individua principalmente due grandi famiglie di \textit{tecniche di fault injection}:
(i) \textsc{Fault Injection Hardware} di cui non ci occupiamo; (ii) \textsc{Fault Injection Software}, è la famgilia  di tecniche che negli ultimi anni ha attirato l'interesse dei ricercatori in quanto tali metodi non richiedono hardware costoso. Inoltre nei contesti in cui il target sia un \textit{applicativo} o, ancora peggio, il \textit{sistema operativo}, costituiscono l'unica scelta.

\noindent
Nel lavoro qui presentato si adotta un approccio \textit{software} che si pone principalmente \textbf{due obiettivi}: 
\begin{enumerate}
    \itemsep-0.3em
    \item La modifica del codice per \textit{casi di studio scelti} volta ad introdurre ridondanza nel sistema tramite la \textbf{duplicazione di tutte le variabili}; 
    \item La realizzazione di un \textbf{ambiente software di fault injection} per simulare l'occorrenza di guasti nel sistema irrobustito e valutarne l'\textit{affidabilità}. Il \textbf{modello di fault} analizzato è il \textit{transient single bit-flip fault}, su cui si basano molti tool di fault injection. 
\end{enumerate}

\noindent 
Il presente documento si pone l'obiettivo di evidenziare i passaggi salienti dell'implementazione delle tecniche di \textit{Software fault tolerance} e dell'\textit{ambiente di fault injection} riportando schemi e \textit{code snippet}  che insieme al codice sorgente -- in \textbf{linguaggio Rust} -- permettono di seguire le diverse fasi del lavoro svolto.\\
Nella \textbf{Sezione \ref{sec:hardened}} viene introdotto un \textbf{set di trasformazioni del codice} alla stregua di parte di quello che viene presentato in \cite{rebaudengo1999soft}. L'implementazione di queste regole è integrata nella realizzazione di un nuovo tipo generico(\texttt{Hardened<T>}) di cui si descrivono gli aspetti chiave. La \textbf{Sezione \ref{sec:CaseStudies}} fornisce un'analisi comparativa del codice non irrobustito rispetto a quello che utilizza le variabili di tipo generico \texttt{Hardened<T>}. Nella \textbf{Sezione \ref{sec:Fault_Env}} viene definita la struttura dell'ambiente di fault injection, mentre le \textbf{Sezioni \ref{sec:FLM}, \ref{sec: Injector}, \ref{sec:analizer}} si occupano di analizzarne i componenti.
