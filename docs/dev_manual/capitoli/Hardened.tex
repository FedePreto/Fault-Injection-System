\section{Software fault-tolerance} \label{sec:hardened}
In generale le tecniche di software fault-tolerance, tendono a sfruttare modifiche di alto livello al codice sorgente in modo da poter rilevare comportamenti irregolari (faults) che riguardano \textbf{sia il codice che i dati}. Qui invece \textit{poniamo l’attenzione esclusivamente su fault che riguardano i dati},  senza peraltro preoccuparci del fatto che questi si trovino in memoria centrale, memoria cache, registri o bus. Al codice target infatti vengono applicate semplici trasformazioni di alto livello che sono completamente indipendenti dal processore che esegue il programma. \\
In questa sezione, dopo una prima analisi, si descrivono le principali caratteristiche e i metodi offerti dal nuovo tipo, in un secondo momento si entra nel dettaglio del linguaggio e si pone l'attenzione all'implementazione della semantica richiesta da \textbf{(R1)-(R3)}.
\subsection{Tre regole per la trasformazione del codice}
Le regole di trasformazione del codice citate sono quelle proposte in \cite{rebaudengo1999soft}. Riportiamo qui quelle mirate al rilevamento di \textbf{errori sui dati}:
\begin{enumerate}
    \itemsep-0.2em
    \item \textbf{Regola \#1}: Ogni variabile \texttt{x} deve essere duplicata: siano \texttt{cp1} e \texttt{cp2} i nomi delle due copie;
    \item \textbf{Regola \#2}: Ogni operazione di scrittura su \texttt{x} deve essere eseguita su entrambe le copie \texttt{cp1} e \texttt{cp2};
    \item \textbf{Regola \#3}: Dopo ogni operazione di lettura su $x$, deve essere controllata la consistenza delle copie \texttt{cp1} e \texttt{cp2}, nel caso in cui non lo siano deve essere sollevato un errore.
\end{enumerate}
Anche i parametri passati a una procedura, così come i valori di ritorno, sono variabili e in quanto tali anche ad esse vanno applicate le stesse trasformazioni. L'implementazione di queste regole -- come spiegato in dettaglio nel paragrafo successivo -- si basano sulla programmazione generica e polimorfismo offerti dal linguaggio Rust. 

\subsection{Il tipo \texttt{Hardened<T>}}
Le tre regole di trasformazione appena esposte sono espletate tramite l'implementazione di un \textbf{nuovo tipo}, che abbiamo denominato \texttt{Hardened<T>}, definito come segue: 

\begin{lstlisting}[language=Rust, style=boxed]
#[derive(Clone, Copy)]
struct Hardened<T>{
    cp1: T, 
    cp2: T
}
\end{lstlisting}
\noindent
Poiché si vuole costruire un nuovo tipo in grado di associare a quanti più tipi standard possibili un dato comportamento, si usa la programmazione generica. In particolare, le due copie \texttt{cp1} e \texttt{cp2} hanno un tipo generico \texttt{T} a cui viene posto l'unico vincolo di essere confrontabile e copiabile.
Al fine di coprire il maggior numero di casistiche possibili in cui il dato viene acceduto in lettura e/o scrittura, sono stati implementati per \texttt{Hardened<T>} un numero significativo di \textbf{tratti della libreria standard}, in particolare: 
\begin{itemize}
    \itemsep-0.3em
    \item \texttt{From<T>}: per ricavare una variabile ridondata a partire da una variabile 'semplice' di tipo T; 
    \item I tratti per le \textbf{operazioni aritmetiche} \texttt{Add, Sub, Mul}. In particolare i primi sono stati implementati anche in \textit{versione mista} \texttt{Add<usize>} e \texttt{Sub<usize>} per semplificare le operazioni di sottrazione tra un \texttt{Hardened<T>} e un valore \textit{literal}; 
    \item I tratti per le \textbf{operazioni di confronto} \texttt{Eq, PartialEq, Ord, PartialOrd}; 
    \item I tratti \texttt{Index} e \texttt{IndexMut}, sotto diverse forme, utili per accedere alla singola copia della variabile  in fase di iniezione e per accedere all'elemento i-esimo di una \textit{collezione} di \texttt{Hardened<T>}.
    \item Il tratto \texttt{Debug} per la visualizzazione personalizzata di informazioni sul nuovo tipo di dato.
\end{itemize}

\noindent
Oltre ai tratti della libreria standard appena elencati,  si è rivelata utile l'implementazione di funzioni personalizzate di cui si riporta una breve descrizione.
\begin{lstlisting}[language=Rust, style=boxed]
impl<T> Hardened<T>{
    fn incoherent(&self)->bool;
    pub fn assign(&mut self, other: Hardened<T>)->Result<(), IncoherenceError>;
    pub fn from_vec(vet: Vec<T>)->Vec<Hardened<T>>;
    pub fn from_mat(mat: Vec<Vec<T>>) -> Vec<Vec<Hardened<T>>>;
    pub fn inner(&self)->Result<T, IncoherenceError>;
}
\end{lstlisting}

\begin{description}
    \item[\texttt{fn incoherent(\&self)->bool}] Funzione privata per rilevare l'incoerenza tra le due copie della variabile irrobustita: in particolare viene utilizzata dai metodi di più alto livello che lavorano con i dati elementari.
    \item[\texttt{pub fn assign(\&mut self, other: Hardened<T>)->Result<(), IncoherenceError>;}]\quad \newline Asserisce all'\textbf{assegnazione} tra due variabili di tipo \texttt{Hardened<T>}. Questa è l'unica operazione che non si può ridefinire in Rust tramite l'implementazione del tratto opportuno, in quanto andrebbe ridefinita l'intera semantica legata al \textbf{movimento e possesso}.
    \item[\texttt{ pub fn from\_vec(vet: Vec<T>)->Vec<Hardened<T>>;} ] Per estrarre collezioni di dati irrobustiti da collezioni di dati elementari. Un ruolo simile è svolto da \newline \texttt{pub fn from\_mat(mat: Vec<Vec<T>>) -> Vec<Vec<Hardened<T>>>;} Queste funzioni sono indispensabili sia per l'implementazione che per l'analisi dei risultati dei casi di studio.
    \item[\texttt{pub fn inner(\&self)->Result<T, IncoherenceError>;}] esegue una sorta di \textit{unwrap} del dato irrobustito, cioè dato un \texttt{Hardened<T>} restituisce il dato \texttt{T} incapsulato a sua volta in un \texttt{Result} in quanto le copie memorizzate possono essere incoerenti (vedi paragrafo dopo).
\end{description}

\section{Regole di trasformazione: implementazione}\label{sec:transf_impl}
In questo paragrafo, tramite l'utilizzo di esempi significativi si presenta a grandi linee l'implementazione del set di trasformazioni che portano al tipo irrobustito. In particolare, dopo aver richiamato la regola, seguirà un esempio di codice con la relativa implementazione.\\

\noindent
\begin{center}
\begin{tikzpicture}
    \node [mybox] (box){%
        \begin{minipage}{.96\textwidth}    
                \begin{center}
                    \large
                \textbf{R1}: ogni variabile \texttt{x} \textbf {deve essere duplicata}: siano \texttt{cp1} e \texttt{cp2} i nomi delle due copie
                \end{center}
        \end{minipage}
    };
\end{tikzpicture}%
\end{center}

\noindent
La realizzazione della prima regola è insita nella definizione del nuovo tipo, in quanto una dichiarazione l'inizializzazione di una variabile di tipo \texttt{Hardened<T>} a partire da un dato elementare, crea una doppia copia del dato stesso. Si veda il seguente esempio: 
\begin{lstlisting}[language=rust, style=boxed]
let mut myvar=15; 
let mut hard_myvar = Hardened::from(myvar);
\end{lstlisting}
Tramite il metodo \texttt{from()} del tratto \texttt{From} infatti vengono popolati i campi \texttt{cp1} e \texttt{cp2} della nuova variabile \texttt{hard\_myvar} nel modo seguente: 

\begin{lstlisting}[language=rust, style=boxed]
impl<T> From<T> for Hardened<T> where T:Copy{
    fn from(value: T) -> Self {
        // Regola 1: duplicazione delle variabili
        Self{cp1: value, cp2: value}
    }
}   
\end{lstlisting}

\noindent
\begin{center}
    \begin{tikzpicture}
        \node [mybox] (box){%
            \begin{minipage}{.96\textwidth}    
                \begin{center}
                    \large
                    \textbf{R2}: ogni \textbf{operazione di scrittura} su \texttt{x} deve essere eseguita su entrambe le copie \texttt{cp1} e \texttt{cp2}
                \end{center}
            \end{minipage}
        };
    \end{tikzpicture}%
\end{center}
Come esempio significativo si consideri il frammento di codice dell'operazione di \texttt{assign()}:

\begin{lstlisting}[language=rust, style=boxed]
pub fn assign(&mut self, other: Hardened<T>)->Result<(), IncoherenceError>{
    //                  [... ]

    //Regola 2: Ogni scrittura deve essere eseguita su entrambe le copie
    self.cp1 = other.cp1;
    self.cp2 = other.cp2;
    Ok(())
}
\end{lstlisting}
Dopo un controllo di coerenza della variabile da assegnare (paragrafo successivo), si scrive sia su una copia che sull'altra.

\noindent
\begin{center}
    \begin{tikzpicture}
        \node [mybox] (box){%
            \begin{minipage}{.96\textwidth}    
                \begin{center}
                    \large
                    \textbf{R3}:  dopo ogni \textbf{operazione di lettura} su $x$, deve essere controllata la consistenza delle copie \texttt{cp1} e \texttt{cp2}, nel caso in cui tale controllo fallisca deve essere sollevato un errore.
                \end{center}
            \end{minipage}
        };
    \end{tikzpicture}%
\end{center}
Per chiarire l'implementazione della terza regola, si riporta un frammento differente della funzione usata in precedenza: 

\begin{lstlisting}[language=rust, style=boxed]
//uso di assign()
let mut a = Hardened::from(4); 
let mut b = Hardened::from(2); 
a.assign(b);  //'a=b'

pub fn assign(&mut self, other: Hardened<T>)->Result<(), IncoherenceError>{
    //Regola 3: lettura, controllo di coerenza, errori
    if other.incoherent(){
        return Err(IncoherenceError::AssignFail)
    }
    // [...]
}
fn incoherent(&self)->bool{ self.cp1 != self.cp2 }
\end{lstlisting}
Usando la funzione \texttt{assign()}, poiché leggo la variabile \texttt{b} è necessario un controllo di consistenza delle due copie, questo è espletato dalla funzione \texttt{incoherent()} che ritorna un booleano. Nel caso in cui questo test non sia passato, si ritorna un \texttt{Err(IncoherenceError)}.

\subsection{Gestione degli errori}
La regola \textbf{R3} richiede che, nel caso il controllo di coerenza fallisca, venga sollevato un errore. Si sono utilizzati principalmente due meccanismi per asserire a questo task:
\begin{enumerate}
    \item Propagazione di un errore di tipo \texttt{IncoherenceError} 
    \item \textbf{Uso della macro \texttt{panic!(...)}}
\end{enumerate}
Il motivo per cui si è dovuto distinguere tra questi due casi è legata alle caratteristiche del linguaggio Rust. In particolare, alcuni tratti della libreria standard permettono -- usando la programmazione generica -- di personalizzare sia il tipo dei dati su cui si opera sia il tipo dei valori di ritorno. In altre situazioni, ad esempi nei tratti associati alle \textit{operazioni di confronto}, non è possibile modificare la \textit{firma dei metodi} e questo è stato il caso in cui si è presentata la necessità di di generare un \texttt{panic!(...)} nel caso di anomalia rilevata. In questo modo abbiamo potuto lasciare invariata la firma dei metodi garantendo la correttezza sintattica.
Di seguito si mostrano due esempi con l'obiettivo di chiarire meglio gli aspetti appena introdotti:
 

\begin{multicols}{2}
\noindent
\textbf{Uso di \texttt{IncoherenceError} }
\begin{lstlisting}[language=Rust, style=boxed]
impl Add<usize> for Hardened<usize>{
  type Output = Result<Hardened<usize>, 
                    IncoherenceError>;
  fn add(self, rhs: usize) -> Self::Output {
    if self.incoherent() {
        return Err(IncoherenceError::AddFail);
    }
    Ok(Self{
        cp1: self.cp1 + rhs,
        cp2: self.cp2 + rhs,
    })
  }
}
\end{lstlisting}
\newcolumn
Si presenta qui l'implementazione del metodo \texttt{add()} del tratto omonimo. La presenza del tipo associato al tratto, permette di non essere vincolati sul tipo di ritorno che quindi è stato personalizzato secondo le nostre esigenze. In particolare poiché l'add come tutte le operazioni di lettura e modifica possono causare errori dovuti all'incoerenza tra le copie interne del dato, si sfrutta l'enumerazione generica \texttt{Result<T,E>} per gestire queste due situazioni. Il tipo \texttt{T} è \texttt{Hardened<T>} mentre il tipo \texttt{E} è quello personalizzato (\texttt{IncoherenceError} descritto in seguito). Un ragionamento analogo vale per tutti i metodi che hanno una struttura simile a quella presentata che abilita l'utilizzo di dati di ritorno personalizzati. 

\newcolumn
\noindent
\textbf{Uso di \texttt{panic!(...)}}
\begin{lstlisting}[language=Rust, style=boxed]
impl<T> Ord for Hardened<T>{
    fn cmp(&self, other: &Self) -> Ordering {
        if other.incoherent(){
            panic!("Ord::cmp");
        }
        self.cp1.cmp(&other.cp1)
    }
}
\end{lstlisting}
In questo caso presentiamo invece un esempio in cui siamo vincolati  sulla scelta del tipo di ritorno e a scatenare dunque un panic. In particolare la funzione \texttt{cmp(...)} del tratto \texttt{Ord} per ovvi motivi vincola il tipo di ritorno ad essere l'enumerativo \texttt{Ordering}. Nel caso in cui il dato letto sia "incoerente", viene sollevato un \texttt{panic!(...)}, in cui il messaggio di errore è di cruciale importanza per l'invio dei risultati dell'iniettore verso l'analizzatore. \\
\hrule
\vspace{0.5cm}
\noindent
Le due casistiche, come si vedrà, nel \textit{processo di fault injection} vengono gestite in modo diverso, mentre per l'analisi il tipo di informazione associato ai due eventi è analogo.

\end{multicols}

\subsubsection{Il tipo di errore \texttt{IncoherenceError}}
Al fine di personalizzare la semantica degli errori e di facilitare il processo di analisi dei risultati, si è pensato di implementare un \textbf{tipo di errore personalizzato} denominato \texttt{IncoherenceError}. Il crate \textbf{\texttt{thiserror}} permette di derivare l'implementazione del tratto \texttt{Error} richiesta da altri meccanismi interni al linguaggio quali la propagazione tramite \textit{question mark operator} e la descrivibilità dell'errore stesso. Il tipo introdotto è un enumerativo: 

\begin{lstlisting}[language=Rust, style=boxed]
#[derive(Error, Debug, Clone)]
pub enum IncoherenceError{
    #[error("IncoherenceError::AssignFail: assignment failed")]
    AssignFail,
    #[error("IncoherenceError::AddFail: due to incoherence add failed")]
    AddFail,
    #[error("IncoherenceError::SubFail: due to incoherence add failed")]
    SubFail,
    #[error("IncoherenceError::MulFail: due to incoherence mul failed")]
    MulFail,
    #[error("IncoherenceError::IndexMutFail: ")]
    IndexMutFail,
    #[error("IncoherenceError::IndexFail: ")]
    IndexFail,
    #[error("IncoherenceError::OrdFail: ")]
    OrdFail,
    #[error("IncoherenceError::PartialOrdFail: ")]
    PartialOrdFail,
    #[error("IncoherenceError::PartialEqFail: ")]
    PartialEqFail,
    #[error("IncoherenceError::InnerFail")]
    InnerFail,
}
\end{lstlisting}
Questo costituisce il tipo \texttt{E} integrato nella variante \texttt{Err(E)} di \texttt{Result}. Con quest'ultimo dettaglio, abbiamo ora il quadro completo delle trasformazioni del codice atte ad introdurre ridondanza nei dati. \\
Nella prossima sezione si introducono i casi di studio, questi costituiscono l'applicazione di tutti i concetti che abbiamo visto finora sull'irrobustimento del codice.

